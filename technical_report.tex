\documentclass[11pt,a4paper]{article}

% ── Packages ──────────────────────────────────────────────────────────
\usepackage[utf8]{inputenc}
\usepackage[T1]{fontenc}
\usepackage{textcomp}
\usepackage{amsmath,amssymb,amsthm}
\usepackage{mathtools}
\usepackage{bm}
\usepackage{booktabs}
\usepackage{array}
\usepackage{multirow}
\usepackage{graphicx}
\usepackage[margin=25mm]{geometry}
\usepackage{enumitem}
\usepackage{algorithm}
\usepackage{algpseudocode}
\usepackage[hidelinks]{hyperref}
\usepackage{cleveref}
\usepackage{xcolor}
\usepackage{float}
\usepackage{caption}
\usepackage{subcaption}

% ── Notation shortcuts ────────────────────────────────────────────────
\newcommand{\RR}{\mathbb{R}}
\newcommand{\NN}{\mathbb{N}}
\newcommand{\ind}{\mathbf{1}}
\DeclareMathOperator{\median}{median}
\DeclareMathOperator{\MAD}{MAD}
\DeclareMathOperator{\EWMA}{EWMA}

\theoremstyle{definition}
\newtheorem{definition}{Definition}[section]

% ── Document ──────────────────────────────────────────────────────────
\title{%
  \textbf{Automated Building Pathology Detection\\
  from Embedded Hygrothermal Sensor Networks}\\[6pt]
  \large Algorithmic Framework and Mathematical Formulation (v6)
}
\author{Senzomatic Technical Report}
\date{February 2026}

\begin{document}
\maketitle

% ══════════════════════════════════════════════════════════════════════
\begin{abstract}
We present a signal-processing pipeline for automated detection of building pathologies
from embedded hygrothermal (HT) sensor arrays installed in wall cavities.
The system ingests four time-series channels---ambient temperature, ambient relative
humidity, cavity relative humidity, and material moisture resistance---and produces
per-device diagnostic reports covering five pathology classes: moisture intrusion,
condensation risk, drying failure, sensor malfunction, and rapid moisture change.
Detection is supported by multi-resolution signal decomposition, time-based differencing,
threshold hysteresis, fleet-seasonal reference profiling, CUSUM change-point analysis,
exponential drying-curve fitting, Hampel-filter spike detection, and MAD-based
installation-level outlier identification.
Each device receives a composite health score on a 0--100 scale.
The pipeline has been validated on 99 sensors across 11 building installations.
\end{abstract}

\tableofcontents
\newpage

% ══════════════════════════════════════════════════════════════════════
\section{Sensor Model and Notation}
\label{sec:notation}

Each sensing device $d$ produces up to four synchronous time-series sampled at
irregular intervals and resampled onto a uniform 5-minute grid
($\Delta t = 300$\,s, i.e.\ 288 samples/day):

\begin{align}
  T^{(d)}(t)   &\in \RR   && \text{ambient temperature [\textdegree C]} \\
  H_a^{(d)}(t) &\in [0,100]  && \text{ambient relative humidity [\%]} \\
  H_c^{(d)}(t) &\in [0,100]  && \text{cavity relative humidity [\%]} \\
  M^{(d)}(t)   &\in [0,100]  && \text{material moisture resistance [\%]}
\end{align}

Devices that expose only temperature and ambient humidity (two-channel devices) are
still analysed for sensor malfunction; the remaining detectors require the full
four-channel set. We write $N$ for the total number of samples after resampling
and $\mathcal{D}$ for the set of all devices in a deployment.

Throughout, we use the shorthand $x_{t}$ to denote the value of series~$x$ at
time index~$t$.  Bold symbols ($\bm{x}$) denote vectors, and calligraphic
letters ($\mathcal{I}$) denote index sets.


% ══════════════════════════════════════════════════════════════════════
\section{Multi-Resolution Signal Decomposition}
\label{sec:preprocessing}

Raw sensor data contains both informative structure and measurement noise.
Different detection tasks require different levels of smoothing: acute events
(intrusion, rapid change, spike detection) benefit from minimal smoothing that
preserves edges, while trend-based tasks (condensation, drying) benefit from
stronger noise suppression. We therefore decompose each channel into three
resolution levels.

\begin{definition}[Multi-resolution decomposition]
For each channel $x \in \{T, H_a, H_c, M\}$ we define:
\begin{enumerate}[label=(\roman*)]
  \item \textbf{Raw signal} $x^{\mathrm{raw}}_t$: the original resampled values, used exclusively for sensor fault detection.
  \item \textbf{Denoised signal} $x^{\mathrm{den}}_t$: a running median filter applied to $x^{\mathrm{raw}}$, preserving edges and step changes while removing isolated spikes.
  \item \textbf{Trend signal} $x_t$: an exponentially weighted moving average (EWMA) applied to $x^{\mathrm{den}}$, producing a smooth monotone-lag estimate of the underlying trend.
\end{enumerate}
\end{definition}

\subsection{Median Filter (Denoised Level)}

The denoised signal is obtained via a symmetric running median of kernel size
$k_{\mathrm{med}} = 7$ (samples):
\begin{equation}
  x^{\mathrm{den}}_t = \median\!\bigl(x^{\mathrm{raw}}_{t-\lfloor k_{\mathrm{med}}/2\rfloor},\;\ldots,\;x^{\mathrm{raw}}_{t+\lfloor k_{\mathrm{med}}/2\rfloor}\bigr).
  \label{eq:median-filter}
\end{equation}
The median filter is order-statistic based, making it robust to isolated
outliers and salt-and-pepper noise while exactly preserving step edges---a
critical property for intrusion detection, where a sudden drop in $M$ must
not be attenuated.

\subsection{EWMA Trend (Trend Level)}

The trend signal is a causal exponentially weighted moving average:
\begin{equation}
  x_t = \alpha\, x^{\mathrm{den}}_t + (1 - \alpha)\, x_{t-1},
  \qquad
  \alpha = 1 - \exp\!\Bigl(-\frac{\Delta t}{\tau_{\mathrm{ewma}}}\Bigr),
  \label{eq:ewma}
\end{equation}
where $\tau_{\mathrm{ewma}}$ is the half-life parameter, set to 6 hours
(i.e.\ 72 samples). The EWMA is strictly causal ($x_t$ depends only on
$x^{\mathrm{den}}_s$ for $s \le t$), ensuring that detection timestamps are
not shifted forward in time---an essential property for real-time deployment.


% ══════════════════════════════════════════════════════════════════════
\section{Time-Based Differencing}
\label{sec:time-delta}

Classical finite-difference operators $\Delta_n x_t = x_t - x_{t-n}$ assume
a perfectly regular sampling grid. In practice, data gaps (sensor downtime,
transmission failures) cause $\Delta_n$ to compare values separated by
irregular physical time spans. We replace fixed-step differences with
time-indexed shifts.

\begin{definition}[Time-based delta]
For a time-series $x$ indexed by timestamps and a target lag $\delta$
(e.g.\ 24\,h, 3\,d, 7\,d, 14\,d):
\begin{equation}
  \Delta_{\!\delta}\, x_t \;=\; x_t - x_{t - \delta},
  \label{eq:time-delta}
\end{equation}
where $x_{t-\delta}$ is obtained by shifting the datetime index by $\delta$
and aligning via the nearest earlier observation. If no observation exists
within the alignment tolerance, the result is $\mathrm{NaN}$ (missing).
\end{definition}

This ensures that ``24-hour drop'' genuinely means 24 hours of elapsed time
regardless of sampling irregularities.


% ══════════════════════════════════════════════════════════════════════
\section{Seasonal Baseline and Fleet Profiling}
\label{sec:seasonal}

Building cavities exhibit pronounced seasonal humidity cycles driven by
ambient temperature and precipitation patterns. Detectors that flag absolute
threshold exceedances alone produce excessive false positives during humid
seasons when \emph{all} devices in a building are elevated. We therefore
construct two reference signals: a per-device seasonal baseline and a
fleet-wide monthly profile.

\subsection{Per-Device Seasonal Baseline}

For each device, we compute a causal rolling baseline over a window of
$W = 30$ days ($W_s = 8{,}640$ samples):
\begin{align}
  \bar{x}_t^{(W)} &= \median\!\bigl(x_s : s \in [t - W_s,\, t]\bigr),
  \label{eq:seasonal-median} \\
  \sigma_t^{(W)}  &= \operatorname{std}\!\bigl(x_s : s \in [t - W_s,\, t]\bigr),
  \quad \sigma_t^{(W)} \ge 1.0,
  \label{eq:seasonal-std} \\
  z_t &= \frac{x_t - \bar{x}_t^{(W)}}{\sigma_t^{(W)}}.
  \label{eq:deviation}
\end{align}
The deviation score $z_t$ measures how many rolling standard deviations the
current value is above its own seasonal norm. A value $z_t > 1.0$ indicates
the device is abnormally elevated relative to its own recent history.

\subsection{Fleet Monthly Profile}

The fleet profile aggregates all devices sharing a cavity humidity channel:
\begin{equation}
  \hat{H}_c^{\,\mathrm{fleet}}(m)
    = \median_{d \in \mathcal{D}}\!\Bigl(\,
        \frac{1}{|\mathcal{T}_m^{(d)}|}\sum_{t \in \mathcal{T}_m^{(d)}} H_c^{(d)}(t)
      \Bigr),
  \qquad m = 1,\ldots,12,
  \label{eq:fleet-seasonal}
\end{equation}
where $\mathcal{T}_m^{(d)}$ is the set of timestamps in calendar month~$m$
for device~$d$. The fleet profile captures the \emph{expected} seasonal
behavior; a device whose cavity humidity exceeds the fleet median by more
than $\delta_{\mathrm{fleet}} = 8$~percentage points is flagged regardless
of its own seasonal deviation.


% ══════════════════════════════════════════════════════════════════════
\section{Detector 1: Moisture Intrusion}
\label{sec:intrusion}

Moisture intrusion occurs when liquid water enters the construction assembly,
causing a simultaneous drop in moisture resistance and rise in cavity humidity.
Detection uses the denoised signals and operates at two time scales.

\subsection{Threshold-Based Detection}

The primary flag combines two conditions via logical conjunction:
\begin{equation}
  F^{\mathrm{intr}}_t =
    \underbrace{\bigl(\Delta_{24\mathrm{h}} M^{\mathrm{den}}_t < -\theta_M^{24}\bigr)
    \wedge
    \bigl(\Delta_{24\mathrm{h}} H_c^{\mathrm{den}}_t > \theta_{H_c}^{24}\bigr)}_{\text{acute event}}
    \;\;\bigvee\;\;
    \underbrace{\bigl(\Delta_{7\mathrm{d}} M^{\mathrm{den}}_t < -\theta_M^{7}\bigr)}_{\text{sustained drop}},
  \label{eq:intrusion-flag}
\end{equation}
where $\theta_M^{24} = 3.0$ (percentage-point drop in moisture resistance
over 24\,h), $\theta_{H_c}^{24} = 8.0$ (percentage-point rise in cavity
humidity over 24\,h), and $\theta_M^{7} = 5.0$ (7-day sustained drop).

An additional noise filter requires that both the moisture drop and cavity
rise exceed a minimum magnitude $\theta_{\min} = 1.0$~pp, eliminating events
where $\Delta = 0$ or the changes are within measurement noise:
\begin{equation}
  \bigl|\Delta_{24\mathrm{h}} M^{\mathrm{den}}_t\bigr| > \theta_{\min}
  \quad\wedge\quad
  \Delta_{24\mathrm{h}} H_c^{\mathrm{den}}_t > \theta_{\min}.
\end{equation}

\subsection{CUSUM Change-Point Detection for Slow Leaks}
\label{sec:cusum}

Threshold-based methods may miss slow, sustained leaks where cavity humidity
rises gradually without ever crossing an acute-event threshold in a single
time step. We supplement with a one-sided CUSUM (Cumulative Sum) detector.

\begin{definition}[One-sided CUSUM]
Let $\{y_t\}$ be a centred signal with reference level $\mu_0$
(estimated from the first 288 samples). The upper CUSUM statistic is:
\begin{equation}
  S_t^{+} = \max\!\bigl(0,\; S_{t-1}^{+} + y_t - \mu_0 - \nu\bigr),
  \qquad S_0^{+} = 0,
  \label{eq:cusum-upper}
\end{equation}
where $\nu > 0$ is the \emph{allowance} (or drift) parameter that controls
sensitivity---only deviations exceeding $\nu$ per sample accumulate.
An alarm is triggered when $S_t^{+} > h$, where $h$ is the decision threshold.
\end{definition}

For moisture intrusion, we apply upward CUSUM to the cavity humidity trend
signal $H_c(t)$ with parameters $\nu = 0.5$ and $h = 5.0$. Upon alarm at
time $t^*$, we examine a confirmation window of $W_{\mathrm{conf}} = 72$~hours
for corroborating evidence:
\begin{equation}
  \exists\, s \in [t^*, t^* + W_{\mathrm{conf}}]:
  \quad M^{\mathrm{den}}_s - \min_{u \in [t^*, s]} M^{\mathrm{den}}_u > \theta_{\min}.
  \label{eq:cusum-confirm}
\end{equation}
If confirmed, the CUSUM alarm region is merged into the intrusion flag via
logical disjunction with \cref{eq:intrusion-flag}.

After each alarm, the CUSUM statistic is reset to zero and the reference
$\mu_0$ is re-estimated from the preceding 288 samples to adapt to the
new regime. A dead-zone of 288 samples between consecutive alarms prevents
redundant detections.

\subsection{Severity Classification}

Detected intrusion episodes are classified into three severity levels based
on the magnitude of the 7-day moisture drop:
\begin{equation}
  \text{severity} =
  \begin{cases}
    \textsc{critical} & \text{if } \Delta_{7\mathrm{d}} M < -2\theta_M^{7}, \\
    \textsc{danger}   & \text{if } \Delta_{24\mathrm{h}} M < -2\theta_M^{24}, \\
    \textsc{warning}  & \text{otherwise}.
  \end{cases}
\end{equation}


% ══════════════════════════════════════════════════════════════════════
\section{Detector 2: Condensation Risk}
\label{sec:condensation}

Sustained high cavity humidity creates risk of condensation and mould growth.
Detection combines absolute threshold exceedance, seasonal deviation, fleet
comparison, and supplementary thermodynamic signals.

\subsection{Threshold Hysteresis}
\label{sec:hysteresis}

Simple threshold crossing $H_c > \theta$ generates excessive toggling when
the signal fluctuates near the threshold. We apply Schmitt-trigger hysteresis
with a band of $\beta = 2.0$~pp:

\begin{definition}[Threshold hysteresis]
The binary state $A_t \in \{0, 1\}$ evolves as:
\begin{equation}
  A_t =
  \begin{cases}
    1      & \text{if } H_c(t) > \theta_{\mathrm{enter}}, \\
    0      & \text{if } H_c(t) < \theta_{\mathrm{exit}}, \\
    A_{t-1} & \text{otherwise},
  \end{cases}
  \qquad
  \theta_{\mathrm{exit}} = \theta_{\mathrm{enter}} - \beta,
  \label{eq:hysteresis}
\end{equation}
with $A_0 = 0$.
\end{definition}

Three severity tiers are defined with the following entry/exit thresholds:

\begin{center}
\begin{tabular}{lccc}
  \toprule
  \textbf{Severity} & $\theta_{\mathrm{enter}}$ [\%] & $\theta_{\mathrm{exit}}$ [\%] & \textbf{Interpretation} \\
  \midrule
  \textsc{warning}  & 80 & 78 & mould risk \\
  \textsc{danger}   & 90 & 88 & condensation likely \\
  \textsc{critical} & 95 & 93 & active condensation \\
  \bottomrule
\end{tabular}
\end{center}

\subsection{Seasonal-Aware Flagging}

For the \textsc{warning} tier, we require both an absolute exceedance
\emph{and} an above-normal seasonal deviation:
\begin{equation}
  F_t^{\mathrm{warn}} = A_t \;\wedge\; (z_t > 1.0),
  \label{eq:condensation-seasonal}
\end{equation}
where $z_t$ is the seasonal deviation from \cref{eq:deviation}. This
suppresses flags during periods when all devices are seasonally elevated.

\subsection{Fleet Seasonal Reference}

A device that is consistently above the fleet median---even if stable by its
own seasonal baseline---may indicate a localised problem. We add a fleet
deviation criterion:
\begin{equation}
  F_t^{\mathrm{fleet}} = A_t \;\wedge\;
    \bigl(H_c(t) > \hat{H}_c^{\,\mathrm{fleet}}(m_t) + \delta_{\mathrm{fleet}}\bigr),
  \label{eq:fleet-flag}
\end{equation}
where $m_t$ is the calendar month of time $t$ and
$\delta_{\mathrm{fleet}} = 8$~pp. The final \textsc{warning} flag is:
\begin{equation}
  F_t^{\mathrm{warn,final}} = F_t^{\mathrm{warn}} \;\vee\; F_t^{\mathrm{fleet}}.
\end{equation}

\subsection{Absolute Humidity Supplement}
\label{sec:vapor}

Relative humidity alone does not capture the total moisture load; identical
RH values at different temperatures correspond to vastly different vapour
concentrations. We compute the absolute humidity (mixing ratio in g/kg) via
the Tetens approximation for saturation vapour pressure:
\begin{align}
  e_s(T) &= 6.1078 \times 10^{\,7.5\,T\,/\,(237.3 + T)}, \label{eq:tetens} \\
  e(T, H_a) &= e_s(T) \cdot \frac{H_a}{100}, \label{eq:vapour-pressure} \\
  w &= \frac{621.97\, e}{P - e}, \qquad P = 1013.25~\text{hPa}, \label{eq:mixing-ratio}
\end{align}
where $T$ is ambient temperature [\textdegree C], $H_a$ is ambient relative
humidity [\%], $e_s$ is saturation vapour pressure [hPa], and $w$ is the
mixing ratio [g\,kg$^{-1}$].

For borderline \textsc{warning} cases (cavity RH within 5~pp of threshold),
high absolute humidity ($w > 14$~g/kg) combined with moderate seasonal
deviation ($z_t > 0.5$) provides supplementary evidence:
\begin{equation}
  F_t^{\mathrm{abs}} = (H_c(t) > \theta - 5) \;\wedge\; (w_t > 14) \;\wedge\; (z_t > 0.5).
\end{equation}

\subsection{Chronic and Recurring Condensation}

\paragraph{Chronic merge.}
When a device spends a large fraction of its data span above a given
threshold, individual episodes are replaced by a single chronic assessment.
Let $p$ be the fraction of time the flag is active:
\begin{equation}
  p = \frac{|\{t : F_t = 1\}|}{N}.
\end{equation}
If $p \ge p_{\mathrm{chronic}}$ and the data span exceeds 6 months, a chronic
problem is emitted. The chronic threshold is severity-dependent:
$p_{\mathrm{chronic}} = 40\%$ for \textsc{warning}, $50\%$ for
\textsc{danger}/\textsc{critical}.

\paragraph{Recurring pattern detection.}
If a device has $\ge 6$ \textsc{warning} episodes spanning $\ge 12$
calendar months (but below the chronic threshold), the fragmented episodes
are consolidated into a single \textsc{recurring} assessment that reports
the total episode count, cumulative duration, and summary statistics.

\subsection{Dew-Point Margin}

As an auxiliary diagnostic, we compute the dew-point margin using the Magnus
formula:
\begin{align}
  \gamma(T, H_a) &= \frac{a\,T}{b + T} + \ln\!\Bigl(\frac{H_a}{100}\Bigr),
  \label{eq:magnus-gamma} \\
  T_d &= \frac{b\,\gamma}{a - \gamma},
  \label{eq:dew-point} \\
  \Delta T_d &= T - T_d,
  \label{eq:dew-margin}
\end{align}
with constants $a = 17.27$, $b = 237.7$~\textdegree C. A small positive
$\Delta T_d$ indicates proximity to condensation conditions.


% ══════════════════════════════════════════════════════════════════════
\section{Detector 3: Drying Failure}
\label{sec:drying}

Post-construction, building cavities should exhibit a monotonically
decreasing humidity profile as trapped moisture evaporates. Failure to dry
(stagnation or reversal) indicates construction defects or inadequate
ventilation. We employ two complementary approaches.

\subsection{Exponential Drying Curve}

Physical drying processes are well-modelled by an exponential decay to an
equilibrium level:
\begin{equation}
  H_c(t) = a + b\,\exp\!\Bigl(-\frac{t}{\tau}\Bigr),
  \label{eq:exp-drying}
\end{equation}
where $a$ is the asymptotic (plateau) humidity, $b$ is the initial excess
above plateau, and $\tau$ is the time constant [days].

\paragraph{Fitting procedure.}
\begin{enumerate}[label=\arabic*.]
  \item Identify the peak cavity humidity $H_c^{\max}$ and restrict the
        fit to the post-peak interval.
  \item Downsample to daily means to reduce noise.
  \item Fit \cref{eq:exp-drying} via nonlinear least-squares
        (Levenberg--Marquardt) with bounds $a \in [0, 100]$,
        $b \in [0, 100]$, $\tau \in [1, 10{,}000]$ days.
  \item Compute the coefficient of determination:
        \begin{equation}
          R^2 = 1 - \frac{\sum_i (y_i - \hat{y}_i)^2}{\sum_i (y_i - \bar{y})^2}.
          \label{eq:r-squared}
        \end{equation}
  \item Accept the fit if $R^2 > 0.5$ and the post-peak span exceeds
        60~days.
\end{enumerate}

\paragraph{Classification.}
\begin{equation}
  \text{severity} =
  \begin{cases}
    \textsc{danger}  & \text{if } \tau > 730~\text{days or } a > 80\%, \\
    \textsc{warning} & \text{if } \tau > 365~\text{days}, \\
    \text{none}      & \text{otherwise}.
  \end{cases}
\end{equation}

A high plateau ($a > 80\%$) is particularly concerning as it indicates the
cavity will never dry to a safe level, regardless of the time constant.

\subsection{Linear Slope Fallback (Moisture Resistance)}

When the exponential fit is not applicable (insufficient data, poor $R^2$,
or moisture resistance channel), we compute quarterly means and fit a linear
trend:
\begin{equation}
  \bar{M}_q = \frac{1}{|\mathcal{T}_q|}\sum_{t \in \mathcal{T}_q} M(t),
  \qquad q = 1, 2, \ldots, Q,
\end{equation}
where $\mathcal{T}_q$ is the set of timestamps in quarter~$q$.
The slope $\hat{\beta}$ is obtained via ordinary least-squares on
$\{(q, \bar{M}_q)\}$:
\begin{equation}
  \hat{\beta} = \frac{\sum_q (q - \bar{q})(\bar{M}_q - \overline{\bar{M}})}
                     {\sum_q (q - \bar{q})^2}.
  \label{eq:linear-slope}
\end{equation}

Classification based on $\hat{\beta}$ and the current 30-day average
$\bar{M}_{\mathrm{cur}}$:
\begin{equation}
  \text{severity} =
  \begin{cases}
    \textsc{danger}  & \text{if } \bar{M}_{\mathrm{cur}} < 50 \text{ and } \hat{\beta} < -0.5~\text{pp/quarter}, \\
    \textsc{warning} & \text{if } \bar{M}_{\mathrm{cur}} < 50 \text{ and } |\hat{\beta}| < 0.3~\text{pp/quarter}, \\
    \text{none}      & \text{otherwise}.
  \end{cases}
\end{equation}


% ══════════════════════════════════════════════════════════════════════
\section{Detector 4: Sensor Malfunction}
\label{sec:sensor}

Sensor faults corrupt downstream analysis. We detect five fault modes
operating on the \emph{raw} (unfiltered) signal.

\subsection{Flatline Detection with Saturation Awareness}

A flatline is defined as a window where the rolling standard deviation drops
to zero:
\begin{equation}
  \sigma^{(W)}_t = \operatorname{std}\!\bigl(x^{\mathrm{raw}}_s : s \in [t - W_f,\, t]\bigr),
  \qquad
  F_t^{\mathrm{flat}} = \ind\!\bigl[\sigma^{(W)}_t < \epsilon\bigr],
  \label{eq:flatline}
\end{equation}
where $W_f = 24$\,h (288 samples) and $\epsilon = 10^{-6}$.

Not all flatlines indicate malfunction. When a humidity sensor saturates at
its physical limits (0\% or 100\%), the signal is genuinely flat but the
sensor is not broken---the environment simply exceeds its range. We classify
flatlines at boundary values $\{0, 100\}$ (within $\pm 0.5$~pp) as
\emph{sensor saturation} (informational) rather than malfunction.

\subsection{Jump Detection with Hampel Enhancement}

Sudden value jumps indicate connection faults or electrical interference.
The primary detector uses first-order differences:
\begin{equation}
  J_t = \ind\!\bigl[|x^{\mathrm{raw}}_t - x^{\mathrm{raw}}_{t-1}| > \theta_J\bigr],
  \label{eq:jump}
\end{equation}
with channel-dependent thresholds ($\theta_J = 10$\textdegree C for
temperature, 25~pp for humidity, 20~pp for moisture). Isolated jumps are
filtered by requiring $\ge 3$ events within a 1-hour window (12 samples):
\begin{equation}
  C_t^{\mathrm{jump}} = \sum_{s=t-11}^{t} J_s \;\ge\; 3.
  \label{eq:jump-cluster}
\end{equation}

\paragraph{Hampel filter enhancement.}
The Hampel filter detects spikes using a robust local dispersion estimate.
For a window of $k = 25$ samples centred at $t$:
\begin{align}
  \tilde{x}_t &= \median\!\bigl(x^{\mathrm{raw}}_s : |s-t| \le \lfloor k/2 \rfloor\bigr),
  \label{eq:hampel-median} \\
  \MAD_t &= \median\!\bigl(|x^{\mathrm{raw}}_s - \tilde{x}_t| : |s-t| \le \lfloor k/2 \rfloor\bigr),
  \label{eq:hampel-mad} \\
  \hat{\sigma}_t &= 1.4826 \cdot \MAD_t,
  \label{eq:hampel-sigma} \\
  O_t &= \ind\!\bigl[|x^{\mathrm{raw}}_t - \tilde{x}_t| > \kappa\, \hat{\sigma}_t\bigr],
  \label{eq:hampel-outlier}
\end{align}
where $\kappa = 3.0$ and the constant 1.4826 ensures Fisher consistency
at the Gaussian model ($\MAD \approx 0.6745\,\sigma$ for $X \sim \mathcal{N}$,
so $1/0.6745 \approx 1.4826$).

To avoid false positives on slowly varying signals where the Hampel filter
detects normal variability, we require that Hampel clusters co-occur with
non-trivial delta activity:
\begin{equation}
  C_t^{\mathrm{spike}} = C_t^{\mathrm{jump}} \;\vee\;
    \Bigl(\sum_{s=t-11}^{t} O_s \ge 3 \;\wedge\;
          \sum_{s=t-11}^{t} \ind\!\bigl[|x_s - x_{s-1}| > \tfrac{1}{2}\theta_J\bigr] \ge 1\Bigr).
  \label{eq:combined-spike}
\end{equation}

\subsection{Out-of-Range Detection}

Values outside the sensor's physical operating range are flagged with
\textsc{critical} severity:
\begin{equation}
  R_t = \ind\!\bigl[x^{\mathrm{raw}}_t < r_{\min} \;\vee\; x^{\mathrm{raw}}_t > r_{\max}\bigr],
\end{equation}
with $r \in \{(-40, 60)~\text{\textdegree C},\; (0, 100)~\text{\%}\}$
depending on channel type.

\subsection{Sensor Drift Detection}

Gradual sensor degradation manifests as a slow shift in the relationship
between cavity and ambient humidity. We monitor the rolling ratio:
\begin{equation}
  \rho_t = \frac{\overline{H_c^{\,\mathrm{raw}}}^{(W_d)}_t}
               {\max\!\bigl(\overline{H_a^{\,\mathrm{raw}}}^{(W_d)}_t,\; 1.0\bigr)},
  \label{eq:drift-ratio}
\end{equation}
where $\overline{(\cdot)}^{(W_d)}$ denotes a rolling mean over $W_d = 14$~days.
A drift alarm is raised when the half-window shift in $\rho$ exceeds
$\kappa_d = 3.0$ times the rolling standard deviation of $\rho$:
\begin{equation}
  D_t = \ind\!\Bigl[
    \bigl|\rho_t - \rho_{t - W_d/2}\bigr| > \kappa_d \cdot \sigma_\rho^{(W_d)}
  \Bigr].
  \label{eq:drift-flag}
\end{equation}

\subsection{Stuck-at-Mid-Value Detection}

A sensor producing near-constant readings around its midpoint (50\%) with
very low variation ($<0.5$~pp over 24\,h) but not exactly zero variation
(which would be caught by the flatline detector) indicates a firmware or
ADC fault:
\begin{equation}
  G_t = \ind\!\Bigl[
    \bigl(\max_{[t-W_f, t]} x^{\mathrm{raw}} - \min_{[t-W_f, t]} x^{\mathrm{raw}}\bigr) < 0.5
    \;\wedge\;
    |x^{\mathrm{raw}}_t - 50| < 10
    \;\wedge\;
    \sigma_t^{(W_f)} > \epsilon
  \Bigr].
  \label{eq:stuck-mid}
\end{equation}


% ══════════════════════════════════════════════════════════════════════
\section{Detector 5: Rapid Moisture Change}
\label{sec:rapid}

This detector targets acute moisture resistance transitions that indicate
active water movement through the construction. It operates on the denoised
moisture signal.

\subsection{Derivative-Based Detection}

The primary flag uses time-based 3-day and 14-day deltas:
\begin{equation}
  F_t^{\mathrm{rapid}} =
    \bigl(\Delta_{3\mathrm{d}} M^{\mathrm{den}}_t < -\theta_R^{3}\bigr)
    \;\vee\;
    \bigl(\Delta_{14\mathrm{d}} M^{\mathrm{den}}_t < -\theta_R^{14}\bigr),
  \label{eq:rapid-flag}
\end{equation}
with $\theta_R^{3} = 4.0$~pp and $\theta_R^{14} = 8.0$~pp. The 14-day
component is disabled when the data span is shorter than 14 days to avoid
semantic errors.

\subsection{CUSUM Enhancement}

Downward CUSUM (\cref{sec:cusum}, direction ``down'') is applied to the
denoised moisture signal to catch sustained gradual drops:
\begin{equation}
  S_t^{-} = \max\!\bigl(0,\; S_{t-1}^{-} + \mu_0 - y_t - \nu\bigr),
  \qquad S_0^{-} = 0.
  \label{eq:cusum-down}
\end{equation}
CUSUM alarm regions are merged into the rapid-change flag via disjunction.

\subsection{Severity Classification}
\begin{equation}
  \text{severity} =
  \begin{cases}
    \textsc{critical} & \text{if } \Delta_{3\mathrm{d}} M < -3\theta_R^{3}, \\
    \textsc{danger}   & \text{if } \Delta_{3\mathrm{d}} M < -2\theta_R^{3}, \\
    \textsc{warning}  & \text{otherwise}.
  \end{cases}
\end{equation}


% ══════════════════════════════════════════════════════════════════════
\section{Episode Extraction and Merging}
\label{sec:episodes}

All detectors produce binary flag series $F_t \in \{0, 1\}$. These are
converted to discrete \emph{episodes} (contiguous intervals) via the
following procedure.

\begin{algorithm}[H]
\caption{Episode extraction and merging}
\label{alg:episodes}
\begin{algorithmic}[1]
\Require Flag series $\{F_t\}_{t=1}^{N}$, merge gap $g$, minimum duration $d_{\min}$
\Ensure List of episodes $\mathcal{E}$
\State Find contiguous runs where $F_t = 1$: intervals $\{[s_i, e_i]\}_{i=1}^{K}$
\State Initialise merged list $\mathcal{M} \gets \{[s_1, e_1]\}$
\For{$i = 2, \ldots, K$}
  \If{$s_i - e_{i-1} \le g$}
    \State Extend last interval: $e_{\text{last}} \gets e_i$ \Comment{bridge short gap}
  \Else
    \State Append $[s_i, e_i]$ to $\mathcal{M}$
  \EndIf
\EndFor
\State $\mathcal{E} \gets \{[s, e] \in \mathcal{M} : (e - s) \ge d_{\min}\}$ \Comment{duration filter}
\State \Return $\mathcal{E}$
\end{algorithmic}
\end{algorithm}

The merge gap $g$ is detector-dependent: 6\,h (default), 21~days (for
\textsc{warning} condensation), 48\,h (for sensor malfunction and rapid
moisture change).

\paragraph{Severity deduplication.}
When multiple severity tiers produce overlapping episodes, lower-severity
episodes that are temporally contained within higher-severity episodes are
removed, ensuring each time interval is assigned at most one severity level
per problem type.


% ══════════════════════════════════════════════════════════════════════
\section{Composite Health Score}
\label{sec:health}

Each device receives a health score $S \in [0, 100]$ that integrates all
detected problems:
\begin{equation}
  S = \max\!\Bigl(0,\;\; 100 - \sum_{p \in \mathcal{P}} w_p \cdot
    \bigl(0.3 + 0.7\, f_p\bigr)\Bigr),
  \label{eq:health-score}
\end{equation}
where $\mathcal{P}$ is the set of detected problems (excluding informational
saturation events), $w_p$ is the severity-dependent weight from
\cref{tab:weights}, and $f_p = \min(d_p / D, 1)$ is the duration factor
($d_p$ = problem duration, $D$ = total data span).

\begin{table}[H]
\centering
\caption{Health penalty weights $w_p$ by problem type and severity.}
\label{tab:weights}
\begin{tabular}{lccc}
  \toprule
  \textbf{Problem type} & \textsc{warning} & \textsc{danger} & \textsc{critical} \\
  \midrule
  Condensation risk    & 5  & 15 & 30 \\
  Moisture intrusion   & 10 & 25 & 40 \\
  Drying failure       & 8  & 20 & 35 \\
  Sensor malfunction   & 3  & 10 & 20 \\
  Rapid moisture change & 8  & 20 & 35 \\
  \bottomrule
\end{tabular}
\end{table}

The factor $0.3 + 0.7 f_p$ ensures that even a brief problem incurs a
minimum 30\% of its weight (presence penalty), while chronic problems
($f_p \to 1$) incur the full weight.

Health grades are assigned as:
\begin{center}
\begin{tabular}{lccccl}
  \toprule
  \textbf{Grade} & A & B & C & D & F \\
  \midrule
  \textbf{Score range} & $[90, 100]$ & $[75, 90)$ & $[50, 75)$ & $[25, 50)$ & $[0, 25)$ \\
  \bottomrule
\end{tabular}
\end{center}


% ══════════════════════════════════════════════════════════════════════
\section{Installation-Level Outlier Detection}
\label{sec:outlier}

Devices within the same building installation share similar environmental
conditions. A device that deviates significantly from its siblings warrants
investigation even if it does not individually trigger any per-device
threshold. We use a robust multivariate approach based on MAD z-scores.

\subsection{Feature Extraction}

For each device $d$, we extract a feature vector $\bm{v}^{(d)} \in \RR^p$:

\begin{center}
\begin{tabular}{llp{7cm}}
  \toprule
  \textbf{Feature} & \textbf{Channel} & \textbf{Definition} \\
  \midrule
  $\mathrm{cavity\_p95}$       & $H_c$ & 95th percentile of cavity humidity \\
  $\mathrm{pct\_above\_80}$    & $H_c$ & Percentage of samples with $H_c > 80\%$ \\
  $\mathrm{pct\_above\_90}$    & $H_c$ & Percentage of samples with $H_c > 90\%$ \\
  $\mathrm{max\_consec\_90}$   & $H_c$ & Maximum consecutive hours above 90\% \\
  $\mathrm{moisture\_p05}$     & $M$   & 5th percentile of moisture resistance \\
  $\mathrm{moisture\_mean}$    & $M$   & Mean moisture resistance \\
  $\mathrm{moisture\_slope}$   & $M$   & Linear slope over last 30 days \\
  \bottomrule
\end{tabular}
\end{center}

\subsection{MAD Z-Scores}

For each feature $j$ and installation $\mathcal{I}$, we compute the Median
Absolute Deviation z-score:
\begin{align}
  \tilde{v}_j &= \median_{d \in \mathcal{I}}\!\bigl(v_j^{(d)}\bigr),
  \label{eq:mad-median} \\
  \MAD_j &= \median_{d \in \mathcal{I}}\!\bigl(|v_j^{(d)} - \tilde{v}_j|\bigr),
  \label{eq:mad-def} \\
  z_j^{(d)} &= \frac{v_j^{(d)} - \tilde{v}_j}{1.4826 \cdot \MAD_j}.
  \label{eq:mad-zscore}
\end{align}

A device is flagged as an outlier on feature $j$ if $|z_j^{(d)}| > 3.0$.
The flagging is directional: high values of humidity features ($z > +3$)
indicate condensation risk outliers, while low values of moisture features
($z < -3$) indicate drying failure outliers.

The MAD-based approach is breakdown-resistant: up to 50\% of devices in an
installation can be anomalous before the median and MAD are corrupted, making
it far more robust than the classical mean/IQR approach.

Installation-level outlier detection requires a minimum of 3 devices per
installation to provide meaningful peer comparison.


% ══════════════════════════════════════════════════════════════════════
\section{Validation Results}
\label{sec:results}

The pipeline was validated on a deployment of 99 devices across 11
building installations, with data spanning up to 40 months.

\subsection{Detection Summary}

\begin{center}
\begin{tabular}{lrrrr}
  \toprule
  & \textsc{critical} & \textsc{danger} & \textsc{warning} & \textbf{Total} \\
  \midrule
  Total problems       & 46  & 189 & 427 & 662 \\
  Devices with issues  & --  & --  & --  & 77  \\
  Clean devices        & --  & --  & --  & 22  \\
  \bottomrule
\end{tabular}
\end{center}

\subsection{Health Score Distribution}

\begin{center}
\begin{tabular}{lccccc|cc}
  \toprule
  \textbf{Grade} & A & B & C & D & F & \textbf{Mean} & \textbf{Median} \\
  \midrule
  \textbf{Count} & 38 & 24 & 24 & 10 & 3 & 76 & 79 \\
  \bottomrule
\end{tabular}
\end{center}

\subsection{Installation-Level Outliers}

The MAD z-score analysis identified 17 outlier observations across the fleet,
including:
\begin{itemize}[nosep]
  \item Device 03c28e06: $\mathrm{cavity\_p95} = 100.0\%$ vs.\ installation
        median $83.6\%$ (MAD $z = 6.8$).
  \item Device 9c3786df: $\mathrm{max\_consec\_90} = 13{,}967$\,h vs.\
        installation median $772.5$\,h (MAD $z = 11.5$).
  \item Device 06e01489: $\mathrm{moisture\_p05} = 7.5\%$ vs.\ installation
        median $12.2\%$ (MAD $z = -9.5$).
\end{itemize}

\subsection{Qualitative Assessment}

Visual inspection of device diagnostic plots confirmed:
\begin{enumerate}[nosep]
  \item Condensation episode boundaries align with visible threshold crossings
        in cavity humidity traces; hysteresis eliminates toggling artefacts.
  \item CUSUM-based moisture intrusion detections correspond to simultaneous
        cavity humidity surges and moisture resistance drops.
  \item Exponential drying curves provide physically meaningful time constants
        (e.g.\ $\tau = 483$~days for a slowly-drying cavity).
  \item Hampel-enhanced spike detection correctly identifies sensor faults
        without generating false positives on normal thermal variability.
  \item Fleet seasonal bands provide context for distinguishing building-wide
        seasonal effects from localised pathologies.
\end{enumerate}


% ══════════════════════════════════════════════════════════════════════
\section{Summary of Parameters}
\label{sec:params}

\begin{table}[H]
\centering
\small
\caption{Complete parameter table.}
\label{tab:params}
\begin{tabular}{llrl}
  \toprule
  \textbf{Parameter} & \textbf{Detector} & \textbf{Value} & \textbf{Unit} \\
  \midrule
  Resample interval & Preprocessing & 5 & min \\
  Median filter kernel & Preprocessing & 7 & samples \\
  EWMA half-life & Preprocessing & 6 & hours \\
  \midrule
  $\theta_M^{24}$ (moisture drop 24\,h) & Intrusion & 3.0 & pp \\
  $\theta_M^{7}$ (moisture drop 7\,d) & Intrusion & 5.0 & pp \\
  $\theta_{H_c}^{24}$ (cavity rise 24\,h) & Intrusion & 8.0 & pp \\
  $\theta_{\min}$ (minimum change) & Intrusion & 1.0 & pp \\
  \midrule
  Condensation WARNING & Condensation & 80 / 78 & \% (enter/exit) \\
  Condensation DANGER & Condensation & 90 / 88 & \% (enter/exit) \\
  Condensation CRITICAL & Condensation & 95 / 93 & \% (enter/exit) \\
  Hysteresis band $\beta$ & Condensation & 2.0 & pp \\
  Fleet offset $\delta_{\mathrm{fleet}}$ & Condensation & 8.0 & pp \\
  Abs.\ humidity threshold & Condensation & 14.0 & g/kg \\
  Chronic \% (WARNING) & Condensation & 40 & \% \\
  Chronic \% (DANGER/CRITICAL) & Condensation & 50 & \% \\
  Recurring min episodes & Condensation & 6 & -- \\
  WARNING merge gap & Condensation & 21 & days \\
  \midrule
  CUSUM threshold $h$ & Intrusion / Rapid & 5.0 & -- \\
  CUSUM drift $\nu$ & Intrusion / Rapid & 0.5 & -- \\
  CUSUM confirmation window & Intrusion & 72 & hours \\
  \midrule
  $\theta_R^{3}$ (3-day drop) & Rapid change & 4.0 & pp \\
  $\theta_R^{14}$ (14-day drop) & Rapid change & 8.0 & pp \\
  \midrule
  Exp.\ drying $\tau$ WARNING & Drying & 365 & days \\
  Exp.\ drying $\tau$ DANGER & Drying & 730 & days \\
  Plateau threshold & Drying & 80 & \% \\
  Min fit span & Drying & 60 & days \\
  \midrule
  Flatline window & Sensor & 24 & hours \\
  Jump threshold (temp) & Sensor & 10.0 & \textdegree C \\
  Jump threshold (humidity) & Sensor & 25.0 & pp \\
  Hampel window $k$ & Sensor & 25 & samples \\
  Hampel threshold $\kappa$ & Sensor & 3.0 & -- \\
  Drift window $W_d$ & Sensor & 14 & days \\
  Drift threshold $\kappa_d$ & Sensor & 3.0 & $\sigma$ \\
  \midrule
  MAD z-score threshold & Outlier & 3.0 & -- \\
  Min devices per installation & Outlier & 3 & -- \\
  Seasonal baseline window & Baseline & 30 & days \\
  \bottomrule
\end{tabular}
\end{table}


% ══════════════════════════════════════════════════════════════════════
\section{Conclusion}
\label{sec:conclusion}

The presented system provides a comprehensive, automated diagnostic pipeline
for building hygrothermal health monitoring. By combining multi-resolution
signal processing, statistical change-point detection, physical drying models,
robust outlier identification, and fleet-level benchmarking, the system
produces actionable per-device and per-installation reports with quantified
severity and health scores.

Validation on 99 sensors across 11 installations demonstrates the system's
ability to correctly distinguish genuine pathologies (moisture intrusion,
chronic condensation, drying failure) from benign seasonal variation and
sensor artefacts, while maintaining low false-positive rates through
hysteresis, seasonal normalisation, and minimum-magnitude filtering.

\end{document}
